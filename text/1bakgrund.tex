\subsection{Bakgrund}
Idag är det många som väcks av ett alarm på morgonen. Antingen av en väckar\-klocka eller en mobiltelefon. Av de som använder alarm är det ungefär hälften som använder {snooze-knappen} (Roitmann, 2017). När knappen trycks ner tystnar larmet och ringer igen efter ett antal minuter. Enligt Torbjörn Åkerstedt, sömnforskare på Karolinska Institutet, kan snoozningen göra att man blir ännu trögare när man vaknar en andra gång (TT, 2013).

För att lösa denna problematik med att folk snoozar istället för att gå upp finns det i dagsläget flera olika typer av alarmklockor. Det finns klockor som skjuter iväg föremål som användaren måste finna, klockor som gömmer sig och klockor som stängs av efter genomfört pussel (BoredPanda, 2011). 
\subsection{Syfte}
Syftet med projektet var att göra en väckarklocka som ser till att användaren verkligen går upp på morgonen. %Detta åstadkomms genom att väckarklockan saknar en snooze-knapp och alarmet stängs inte av förrän en klarar av att svara korrekt vilket en måste vakna för. (denna mening kommer att skrivas om och delas upp)
 
 \subsection{Frågeställningar}
\begin{itemize}
    \item {Hur skapas en smart väckarklocka?}
    \item {Hur ska en väckarklocka se till att vi verkligen går upp?}
    %\item {Hur hjälper en väckarklocka i vardagen?}
\end{itemize}

\subsection{Mål}
Målet är att skapa en väckarklocka som ser till att personen går upp när klockan ringer genom att användaren måste besvara en fråga för att tysta alarmet. Det ska vara lätt att kunna lägga till nya frågor och det ska vara intuitiva kontroller för användaren, exempelvis att ställa in alarmtiden och besvara frågor. 

